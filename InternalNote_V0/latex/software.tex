\section{Software}
\label{sec:Software}

The picture of the camera is supposed to be analysed with a pattern recognition software. This insures that no person has to be present during winding time supervising the winding process. For this purpose a pattern recognition software based on the open source library \emph{OpenCV} has been developed. The main idea was to apply a background substraction of the frame at the position of the current fiber such as gaussian mixture models \cite{MOG01,MOG204,GMG12}. This results in a binary image that, besides fluctuations that can be thresholded, resembles if there was a notable change in the picture, i.e. the current fiber not being in place.\\

This method can't be applied to the whole camera frame due to expected vibrations of the camera and a possibly unbalanced winding wheel which would trigger the background substraction over the threshold. Therefore the approximate position of the current fiber and a window around that position need to be estimated. This is done on the basis of a hough transformation into the parameter space of circles \cite{Yuen90} that gives a set of found circles. The reason why this can't directly be used to assure the right positon of the current fiber is, as can later be seen, that it is difficult to find and to find only those circles that correspond to a fiber on the winding wheel. In fact there are much more circles found in the frame than there are fibers on the wheel and these need to be filtered somehow. In the current implementation this is easily be done through taking the mean position of all circle centers and taking only those circles into account that lie in a small window around that position. But another implementation including e.g. clustering could be applied here. Given the filtered circles and say they all correspond to a fiber on the winding wheel the current fiber window and the foreground image can be computed.\\

The problem arises that there is no given point to place the window relative to. Placing it relative to the outermost found circle will more or less result in an error once this circle isn't found in one of the frames. The solution here is measuring the position und the distances of the fibers over an initial range of frames, i.e an initial period of time, while after that time the video feed gets stabilized, i.e. it gets moved back according to an estimated movement between two frames. This is done employing the feature detector described in \cite{ShiTomasi94} to get prominent features that are trackable well and track them with the pyramidial Lucas-Kanade feature tracker \cite{LK81,Bouguet02PyrLK}. The matching pairs of feature points can then be used to estimate the movement between two frames, which is done using a simple least squares algorithm. After that the camera feed should be stabilized and the current fiber should always appear to be locked inside the current fiber window.\\

The goal is to measure the signal inside the computed foreground images or combinations of them, which are binary images that can be normed, and trigger an alarm and possibly bring the winding process to a stop if the signals rise above a given threshold.
